\documentclass{article}
\usepackage[T2A]{fontenc}
\usepackage[utf8]{inputenc}
\usepackage[russian]{babel}
\usepackage[a4paper, left=2cm, right=2cm, top=2cm, bottom=2cm]{geometry}
\setlength{\parskip}{0pt}    
\setlength{\parindent}{0pt}   
\title{Документация библиотеки geometric\_lib}
\author{Курбатов Максим Андреевич M3112}

\date{\today}
\begin{document}
\maketitle
\begin{center}
\large Преподаватель: Жуйков Артём\\
\end{center}

\newpage

\tableofcontents
\newpage
\section{Что включает в себя geometric_lib?}
\\
\largeДанная библиотека содержит 4 файла, написанных на языке Python:\\
\\
\large circle.py - нахождение периметра и площади круга\\
\\
\large square.py - нахождение периметра и площади квадрата\\
\\
\large triangle.py - нахождение периметра и площади треугольника\\
\\
\large calculate.py - нахождение периметра и площади выбранной вами фигуры\\

\section{Каталог geometric_lib}

\subsection{Файл 1: circle.py}
\subsubsection{Описание}
Этот файл содержит функции для вычисления площади и периметра круга.

\subsubsection{Исходный код}
\begin{verbatim}
import math

def area(r):
    """Вычисляет площадь круга по формуле: 
    S = π * r^2, где r - радиус.
    Параметр:
    r - радиус круга.
    Возвращает площадь круга."""
    return math.pi * r * r

def perimeter(r):
    """Вычисляет периметр круга по формуле: 
    P = 2 * math.pi * r, где r - радиус.
    Параметр:
    r - радиус круга.
    Возвращает периметр круга."""
    return 2 * math.pi * r
\end{verbatim}

\subsubsection{Метод работы}
Функция \texttt{area(r)} вычисляет площадь круга, используя формулу \( S = \pi r^2 \). Она принимает один аргумент \( r \) (радиус) и возвращает значение площади. 

Функция \texttt{perimeter(r)} вычисляет периметр круга по формуле \( P = 2\pi r \), также принимая радиус \( r \) и возвращая периметр.

\newpage

\subsection{Файл 2: square.py}
\subsubsection{Описание}
Этот файл содержит функции для вычисления площади и периметра квадрата.

\subsubsection{Исходный код}
\begin{verbatim}
def area(a):
    """Вычисляет площадь квадрата по формуле: 
    S = a^2, где a - длина стороны.
    Параметр:
    a - сторона квадрата.
    Возвращает площадь квадрата."""
    return a * a

def perimeter(a):
    P = 4 * a
    return 4 * a
\end{verbatim}

\subsubsection{Метод работы}
Функция \texttt{area(a)} вычисляет площадь квадрата по формуле \( S = a^2 \), где \( a \) — длина стороны квадрата. Она принимает один аргумент \( a \) и возвращает значение площади.

Функция \texttt{perimeter(a)} находит периметр квадрата, используя формулу \( P = 4a \). Она также принимает сторону \( a \) и возвращает периметр.

\newpage

\subsection{Файл 3: triangle.py}
\subsubsection{Описание}
Этот файл содержит функции для вычисления площади и периметра треугольника.

\subsubsection{Исходный код}
\begin{verbatim}
def area(a, b, c):
    """Вычисляет площадь треугольника по формуле:
    S = (a + b + c) / 2, где a, b, c - длины сторон.
    Параметры:
    a, b, c - стороны треугольника.
    Возвращает площадь треугольника."""
    return (a + b + c) / 2

def perimeter(a, b, c):
    """Вычисляет периметр треугольника по формуле:
    P = a + b + c, где a, b, c - длины сторон.
    Параметры:
    a, b, c - стороны треугольника.
    Возвращает периметр треугольника."""
    return a + b + c
\end{verbatim}

\subsubsection{Метод работы}
Функция \texttt{area(a, b, c)} использует формулу \((a + b + c) / 2\) для вычисления площади треугольника. Параметры \( a \), \( b \) и \( c \) — длины сторон треугольника, а возвращаемое значение — площадь.

Функция \texttt{perimeter(a, b, c)} вычисляет периметр треугольника, складывая длины всех трех сторон. Она принимает три параметра \( a \), \( b \) и \( c \) и возвращает периметр.

\newpage

\subsection{Файл 4: calculate.py}
\subsubsection{Описание}
Этот файл импортирует модули для работы с фигурами и содержит функции для вычисления площади и периметра выбранной фигуры.

\subsubsection{Исходный код}
\begin{verbatim}
import circle
import square

figs = ['circle', 'square']
funcs = ['perimeter', 'area']
sizes = {}

def calc(fig, func, size):
    """Выполняет расчет площади или периметра выбранной фигуры.
    Параметры:
    fig (str) - имя фигуры ('circle' или 'square').
    func (str) - функция для вычисления ('perimeter' или 'area').
    size (list) - размеры, необходимые для вычисления.
    Возвращает результат расчета."""
    assert fig in figs
    assert func in funcs
    result = eval(f'{fig}.{func}(*{size})')
    print(f'{func} of {fig} is {result}')

if __name__ == "__main__":
    func = ''
    fig = ''
    size = list()

    while fig not in figs:
        fig = input(f"Enter figure name, avaliable are {figs}:\n")

    while func not in funcs:
        func = input(f"Enter function name, avaliable are {funcs}:\n")

    while len(size) != sizes.get(f"{func}-{fig}", 1):
        size = list(map(int, input("Input figure sizes separated by space, 1 
        for circle and 
        square\n").split(' ')))

    calc(fig, func, size)
\end{verbatim}

\subsubsection{Метод работы}
Функция \texttt{calc(fig, func, size)} выполняет расчет площади или периметра для заданной фигуры. Она принимает имя фигуры \( fig \) (например, "circle" или "square"), имя функции \( func \) (например, "area" или "perimeter") и список \( size \) с необходимыми параметрами.

Функция использует \texttt{eval()} для динамического вызова соответствующей функции из модуля фигуры. Ввод значений осуществляется через стандартный ввод.
\newpage
\section{Хэши последних коммитов}


\subsection{Хэши}
\\
L-05: Update Docs. Add user agreement info\\
\\
commit 438b89a1dfc58d90e9036fe431771427965cd1ff\\
Author: Danny <bublikplushka@yandex.ru>\\
Date:   Mon Apr 19 15:12:19 2021 +0300\\
\\
    L-05: Add user agreement\\
\\
commit 6adb96248a4d00d3bea13bd95d78ef52352cd1b4\\
Author: smartiqa <info@smartiqa.ru>\\
Date:   Thu Mar 4 14:55:29 2021 +0300\\
\\
    L-03: Docs added\\
\\
commit 30494317cde4419be779c14306561e0eaa78b88b (origin/feature)\\
Author: Daniil.K <dlkay@yandex.ru>\\
Date:   Tue Mar 30 17:36:09 2021 +0300\\
\\
    L-04: Add rectangle.py\\
\\
commit b5b0fae727ca72c317c383b39c0af73d6adcd81c (origin/develop)\\
Author: Daniil.K <dlkay@yandex.ru>\\
Date:   Tue Mar 30 18:02:23 2021 +0300\\
\\
    L-04: Update docs for calculate.py\\
\\
commit d76db2ac7f69cc920ae2e6f669fb0671a7fa7d71\\
Author: Daniil.K <dlkay@yandex.ru>\\
Date:   Tue Mar 30 17:57:42 2021 +0300\\
\\
    L-04: Add calculate.py
\\
commit 51c40ebfd0e0b65f52fe5e54740cbb038e492db3\\
Author: smartiqa <info@smartiqa.ru>\\
Date:   Fri Mar 26 14:52:26 2021 +0300\\
\\
    L-04: Doc updated for triangle
\\
commit d080c7888b81955bad2ed78d58ad910526b5132a\\
Author: smartiqa <info@smartiqa.ru>\\
Date:   Fri Mar 26 14:48:39 2021 +0300\\
\\
    L-04: Triangle added\\
\\
commit d078c8d9ee6155f3cb0e577d28d337b791de28e2 (origin/main, origin/HEAD, main)\\
Author: smartiqa <info@smartiqa.ru>\\
Date:   Thu Mar 4 14:55:29 2021 +0300\\

    L-03: Docs added\\
\\
commit 8ba9aeb3cea847b63a91ac378a2a6db758682460\\
Author: smartiqa <info@smartiqa.ru>\\
Date:   Thu Mar 4 14:54:08 2021 +0300\\
\\


\end{document}
